\documentclass[a4paper,12pt]{article}
\usepackage{graphicx}
\usepackage{times}\fontfamily{ptm}\selectfont
\usepackage[activeacute,english]{babel}
\usepackage{t1enc}
\usepackage{amsmath}
%\setlength{\textheight}{9.1in}
%\setlength{\textwidth}{6.0in}
%\setlength{\topmargin}{-0.18in}
%\setlength{\oddsidemargin}{0in}
%\pagestyle{plain}
\usepackage{longtable} %fixar l�nga tabeller �ver flera sidor
\usepackage{supertabular}
\usepackage{tabularx} %fixar l�nga tabeller
\usepackage{multirow} %fixar nested tables

\newenvironment{newminipage}{\begin{small} \begin{minipage}}{\end{minipage}\end{small}}

\pagenumbering{arabic}
\def\myname{\textbf{Martin Nervall}}

\newcommand{\HR}{\rule{1em}{.4pt}}
\newcommand{\sctn}[1]{\section{\large #1}}
\newcommand{\subsctn}[1]{\subsection{#1}}
\newcommand{\subsubsctn}[1]{\subsubsection{#1}}
\renewcommand{\theenumi}{\alph{enumi}}
\renewcommand{\theenumii}{\arabic{enumii}}
\newcommand{\dGb}{\ensuremath{\Delta G_{\rm bind}}}
\newcommand{\dGel}{\ensuremath{\Delta G_{\rm el}}}
\newcommand{\dGvdw}{\ensuremath{\Delta G_{\rm vdW}}}
\newcommand{\qdyn}{\texttt{Qdyn5}}
\newcommand{\qprep}{\texttt{Qprep5}}
\newcommand{\qcalc}{\texttt{Qcalc5}}
\newcommand{\qfep}{\texttt{Qfep5}}
\newcommand{\q}{\texttt{Q}}
\newcommand{\grep}{\texttt{grep}}
\newcommand{\sed}{\texttt{sed}}
\newcommand{\runsh}{\texttt{run{\_}Q.sh}}
\newcommand{\pymol}{\texttt{PYMOL}}
\newcommand{\chemscore} {\texttt{ChemScore}}
\newcommand{\xscore} {\texttt{X\-Score}}

\title{4. Scoring of nine ligands in p450Cam}
\author{Martin Alml\"{o}f, Martin And\'er, Sinisa Bjelic, Jens Carlsson, \\ Hugo Guti\'errez de Ter\'an, Martin Nervall, Stefan Trobro and Fredrik \"{O}sterberg}
\date{18th August 2005}
\begin{document}

\maketitle
\tableofcontents
\newpage
\renewcommand{\thefigure}{\arabic{figure}}

\section{Scoring Functions}

\subsection{Theoretical background}
Simple functions for estimating free energies of binding have
become popular tools in the area of computer aided drug design.
The functions are characterized by high speed compared to the more
accurate methods of FEP and LIE. The high speed enables
utilization in docking applications where large conformational
searches require repeated estimations of the free energy of
binding to find the most probable conformation. If the docking
procedure is fast enough large virtual libraries can be screened
for binders towards a target receptor.

Scoring functions can in general be divided into several
categories depending on their origin.

\begin{itemize}
\item Empirical scoring functions, derived from parameterizations on
experimental binding data.

\item Knowledge based scoring functions, derived from distance to
distance information in available 3D protein-ligand complexes.

\item Other scoring functions such as force field based and
quantum chemically based.
\end{itemize}

\noindent In this practical we will focus on the empirically based
functions.


\subsubsection{Empirical functions}
Empirical functions rely on a (usually) linear combination of
several contributing terms, e.g.
        \begin{equation} \label{emp_gen}
        \Delta \dGb = \sum_{i} \alpha_i \Delta G_{i}.
        \end{equation}
The $\Delta G_i$'s denote the binding contribution from different
types of interactions and are determined for a large number of
complexes for which {\dGb} is known experimentally. From this, the
$\alpha_i$'s are determined by regression so as to minimize the
prediction error.

The assumption is that the total binding energy can be expressed
as a linear combination of terms, each corresponding to one type
of interaction, and that the contribution from each term can be
determined computationally. Some common types of interactions
considered by empirical scoring functions are:
    \begin{itemize}
      \item{van der Waals interactions}
      \item{hydrogen bonding}
      \item{ionic and metallic interactions}
      \item{lipophilic interactions}
      \item{entropic contributions}
    \end{itemize}


\subsection{ChemScore}
The ChemScore function was published by Eldridge \textit{et al} in
1997 and was calibrated on a set of 82 protein-ligand complexes.
The goal was to develop an empirical function using simple and
physically interpretable terms:
      {\setlength\arraycolsep{2pt}
      \begin{eqnarray}
      \Delta G_{binding} & = & \Delta G_{H-bond} + \Delta G_{metal} + \Delta G_{lipophilic} + \nonumber\\
      & & {}+\Delta G_{deformation} + \Delta G_0.
      \end{eqnarray}}
where $\Delta G_{H-bond}$ considers the contributions from
H-bonds, $\Delta G_{metal}$ is a metal interaction term, $\Delta
G_{lipophilic}$ handles the vdW contribution, $\Delta
G_{deformation}$ is a penalty term for entropic loss and $\Delta
G_0$ is a constant offset.


\subsection{X-Score}
X-Score is an empirical scoring function published by Wang
\textit{et al} in 2002. It was calibrated on a set of 200
protein-ligand complexes in the Protein Data Bank (PDB).

The binding contributions considered by X-Score are van~der~Waals
interactions, lipophilic interactions, hydrogen bonding and
deformation effects. Thus, X-Score takes the following form:
      {\setlength\arraycolsep{2pt}
      \begin{eqnarray}
      \Delta G_{binding} & = & \Delta G_{vdw} + \Delta G_{H-bond} + \Delta G_{lipophilic} + \nonumber\\
      & & {}+\Delta G_{deformation} + \Delta G_0,
      \end{eqnarray}}
where $\Delta G_0$ is a regression constant including an average
of the rotational and translational entropic contributions.

X-Score is a consensus scoring function; three different
algorithms for computing the lipophilic effect are used --- each
generating a separate scoring function. The final score is
computed as the average of these three outputs.

X-score and ChemScore are very similar when you consider the terms
involved, but there are substantial differences in the details of
calculating the corresponding terms.


%*****************************************************
% Practical stuff below
%*****************************************************

\section{Practicals}

The aim of this session is to get acquainted with two of the
scoring functions implemented in the Q5 package, ChemScore and
X-Score. We will use a set of nine inhibitors of the enzyme
p450Cam. To make the practical more interesting we will
investigate the difference between scoring a single snapshot and
scoring a short molecular dynamics trajectory of 50 ps. \\

\noindent Log on to your account if you are not already logged on.
Move to the "SCORE" directory:\\
 \texttt{cd SCORE}

\noindent The first ligand we will score interactively, the rest
is already scripted and fast to run.

\subsctn {ChemScore}

Move to the first ligand directory called "adm" and start {\qcalc}
by
typing\\
 \texttt{cd adm} \\
 \qcalc %Kommer de att ha exec i s�kv�g??

\noindent The first thing the program asks for is the topology
file, which should be a familiar concept by now. In this case the
file is called \texttt{adm.top}

\noindent Now you have come to the stage where you should choose
what calculation to make. Write \texttt{chemscore} and press
enter.

\noindent The program will prompt you for a mask. When you want to
score a trajectory the mask specified here should be the same as
the trajectory was created with. In our case all trajectories were
created with the atom mask \texttt{solute}. End the mask
definition with \texttt{end}.

\noindent The next question you are faced with is whether you want
to score the initial topology or if you want to score a
trajectory. Write \texttt{yes} to score the topology first.

\noindent The fep-file is called \texttt{lig.fep} for all the
ligands. Note that the files are different for the
different ligands even though they have the same name.

\noindent Finally you should define the parameter file containing
parameters for {\chemscore} with the force field Oplsaa. The file
is
in:\\
\texttt{../../Q5/chemscore\_oplsaa.prm
}

\noindent Now that you have defined the calculation you can start
it by typing \texttt{go}. The score of the initial conformation,
i.e. the crystal structure, should appear within a few seconds
(NOTE: ChemScore gives the result in kJ/mol).

As you can see the program is still running. It is waiting for you
to define a trajectory file to continue the calculation on. Type
\texttt{dc1.dcd} and watch the computer work. When you have scored
the whole trajectory of 50 ps with snapshots every 2 ps you have
25 scoring values. Calculate the mean value by typing
\texttt{mean} and finally exit the program by giving the
\texttt{end} command.

\subsctn {X-Score}

Next we will take a look at {\xscore}. Start {\qcalc} again and
define the same topology as before. Now choose the option
\texttt{xscore}. The procedure is just like in the {\chemscore}
case until you are prompted for \texttt{Cofactor}. p450Cam has a
cofactor, a heme group, which we must specify. Define it by typing
\texttt{restype=HEM}. End the definition with a blank line.

\noindent The program will now ask for a code corresponding to the
force field the system is represented in. We are using Oplsaa in
this molecular system and the code to give {\xscore} is
\texttt{qoplsaa}.

\noindent Finally you should define the path to a file with input
parameters for {\xscore}. Enter the file name\\
\texttt{../../Q5/xscore\_default.input}\\

\noindent As before, start the calculation with \texttt{go} and
wait until the score for the initial conformation appears (NOTE: 
X-Score gives the result as p$K_{d}$). Then
type \texttt{dc1.dcd} to score the trajectory, \texttt{mean} to
get the average and \texttt{end} to stop the program.

\subsctn {Scoring nine ligands}

In order to get some statistics for the comparison of scoring a
snapshot versus a molecular dynamics trajectory we will now score
a total of nine ligands in p450Cam. Start the scoring by going
back to the "SCORE" directory and run the shell script
\texttt{score\_all.sh}:\\
\texttt{cd ..} \\
\texttt{csh score\_all.sh}

\noindent The script will perform the same scoring you just did
with the \texttt{adm} ligand for all nine ligands. Thus it will
run both {\chemscore} and {\xscore} and write the result to
\texttt{chemScore.log} and  \texttt{XScore.log} for the respective
scoring function.

\noindent When the script has finished you can view the result in
\texttt{gnuplot} by running the script \texttt{plot\_all.sh} which
will extract data from the logfiles. The script will take the
average of the {\chemscore} and the {\xscore} value for the plot.

\noindent How do the points corresponding to the scored
trajectories fit the experimental values? Compare them to the
values from the scored crystal structures. Which method gives the
best ranking? You can also compare the numbers if you open the
file \texttt{plot.txt} in \texttt{emacs}. The first column is the
experimental values, the second is the scored crystal structures
and the third is the average from the trajectories.



%Finns inte mask info i dcd:n?? Jag tror n�stan det. Kolla!!!

%%De ska scora tre funktioner, p� nio ligander, p� tv� olika s�tt (topologi och sampl)




















\end{document}
